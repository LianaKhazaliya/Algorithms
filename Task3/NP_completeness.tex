\documentclass[10pt]{article}

%\pagestyle{plain}
\textwidth=16cm
\textheight=24.0cm
\oddsidemargin=0.3cm
\evensidemargin=0.5cm


\usepackage[utf8]{inputenc}
\usepackage[russian]{babel}
\usepackage{amssymb,amsmath}
\usepackage{fancyhdr}
\usepackage[yyyymmdd,hhmmss]{datetime}
\usepackage{graphicx}
\usepackage[a4paper,bmargin=2.5cm]{geometry}

\voffset=0pt
\headheight=0pt
\headsep=0pt



\begin{document}
\def\chap#1#2{\ \\ {\large\bf#1 \ | \ \tt\scshape#2} \par}

{\bf
\begin{flushright}
\small{Задание №3 --- <<NP-completeness>> \\ Лиана Хазалия, 3 курс 2 группа\\  \today, \currenttime}
\end{flushright}}
\ \\
\rm
{\large\texttt{Задача 99-1 [NP-Complectness]}}
\\
\textsc{Самый длинный путь}
\ \\[0.2cm]
\textit{Задан граф $G=(V,E)$ и положительное число $K\leq |V|$. Имеется ли в $G$ простой путь (то есть путь, не проходящий дважды ни через одну вершину), состоящий не менее чем из $K$ рёбер?}
\ \\[0.2cm]
\textbf{Решение.}  

\medskip\par Так как длина гамильтонова пути равна $|V|-1$ (проходит по всем вершинам единожды, а также включает каждое из рёбер не более одного раза), то при нахождении гамильтонова пути решаем поставленную задачу для $K=|V|-1$. То есть известная задача UHamPath суть частный случай предложенной, следовательно, предложенная $NP$-сложна.
\medskip\par Рассматриваемая задача принадлежит $NP$, так как для заданного графа $G=(V, E)$ сертификат имеет вид последовательности, состоящей из вершин множества $V'\subset V$, образующих простой путь, $|V'|=K+1$, где $K$ --- количество рёбер в искомом простом пути. В алгоритме верификации проверяется, что в эту последовательность каждая вершина из $V'$ входит единожды и что каждая пара воследовательных вершин соединена ребров. Подобная проверка выполняется за полиномиальное время. 
\medskip\par Из того, что UHamPath $\leq_P$ LongestPath и что LongestPaht $\in$ $NP$, cледует $NP$-полнота предложенной задачи LongestPath.
\ \\[0.5 cm]
{\large\texttt{Задача 100-10 [NP-Complectness]}}
\\
\textsc{Неэквивалентность регулярных выражений, не содержащих звёздочек}
\ \\[0.2cm]
\textit{Заданы два не содержащие звёздочек регулярных выражения $E_1$ и $E_2$ в конечном алфавите $\Sigma$. Такое выражение определяется следующим образом:
\begin{enumerate}
\item любой символ $\sigma$ алфавита $\Sigma$ есть не содержащее звёздочек регулярное выражение,
\item если $e_1$ и $e_2$ --- два не содержащие звёздочек регулярных выражения, то и слова $e_1e_2$ и $(e_1 \vee e_2)$ также не содержащие звёздочек регулярные выражения.
\end{enumerate}
Верно ли, что $E_1$ и $E_2$ представляют различные языки в алфавите $\Sigma$? (Язык, представляемый символом $\sigma\in\Sigma$, если $\{\sigma\}$, а если $e_1$ и $e_2$ представляют соответственно языки $L_1$ и $L_2$, то $e_1e_2$ представляет язык $\{xy:\,x\in L_1,\,y\in L_2\}$, а $(e_1\vee e_2)$ представляет язык $L_1\cup L_2$.)
}

\ \\[0.5 cm]
{\large\texttt{Задача 7 [NP-ISSUES]}}
\ \\[0.2cm]
\textit{
Так как задача о трёхмерном сочетании является $NP$-полной, естественно ожидать, что аналогичная задача о четырёхмерном сочетании будет хотя бы не менее сложной. Определим четырехмерное сочетание следующим образом: для заданных множеств $W$, $X$, $Y$ и $Z$, каждое из которых имеет размер $n$, и набора $C$ упорядоченных четверок в форме $(w_i, x_j, y_k , z_l)$ существуют ли $n$ четверок из $C$, среди которых никакие два не имеют общих элементов?
\\ \
Докажите, что задача о четырехмерном сочетании является $NP$-полной.
}
\ \\[0.2cm]
\textbf{Решение. (не оттуда, не туда)} Пусть есть 4-дольный граф $G(V,E)$ на $4n$ вершинах (в каждой доли $n$ вершин), в котором можно выделить $m$ подграфов $K_4$. Далее отождествим каждую 4-клику, у которой нет двух вершин из одной доли, с новой вершиной, таким образом получим новый граф $G'$. Новые две вершины инцидентны тогда и только тогда, когда у подграфов, с которыми они были отождествлены, пересечение множеств вершин не пусто.
\medskip\par
Если в таком новом графе $G'$ существует независимое множество вершин мощности $n$, то существует и решение у нашей задачи, потому что выделив $n$ вершин в $G'$, не имеющих общих ребер, выделили $n$ 4-клик исходного графа $G$, у которых нет общих элементов и у которых по одной вершине в каждой из четырёх доль в силу проведенного ранее отождествления. А значит, нашли решение исходной задачи. Если же независимого множества можности $n$ не существует, то какие бы $n$ 4-клик с вершинами в попарно различных долях ни были бы выбраны в исходном графе, какое-то из попарных пересечений множеств вершин не пусто, так как соответствующее множество вершин в $G'$ не является независимым.
\ \\[0.1cm]
По решению полученной задачи о поиске независимого множества в графе $G'$ легко восстановить разбиение исходного множества на четверки, элементы которых принадлежат попарно различным долям, а также нет общих.
\ \\[0.5 cm]
{\large\texttt{Задача 16 [NP-ISSUES]}}
\ \\[0.2cm]
\textit{Рассмотрим задачу характеристики множества по размерам его пересечений
с другими множествами. Имеется конечное множество $U$ размера $n$, а также
набор $A_1, \dots, A_m$ подмножеств $U$. Также заданы числа $c_1, \dots, c_m$. Вопрос звучит так: существует ли такое множество $X\subset U$, что для всех $i = 1, 2, \dots, m$ мощность $X \cap A_i$ равна $c_i$? Назовем его задачей выведения пересечений с входными данными $U$, $\{A_i\}$ и $\{c_i\}$.
\\ \
Докажите, что задача выведения пересечений является $NP$-полной.
}
\ \\[0.2cm]
\textbf{Решение.} 

\par Рассмотрим задачу \textsc{Hitting Set}: дано множество $A=\{A_i\}_{i=\overline{1, m}}$ подмножеств множества $U$ и число $k\in\mathbb{Z}^+$, существует ли такое подмножество $X\subset U$, что $|X|\leq k$ и что для любого $i=\overline{1, m}$ выполнено $|A_i\cap X| = 1$?
\medskip\par
\textbf{Лемма.} Задача \textsc{Hitting Set} $NP$-полна.

\medskip\par 
Чтобы доказать, что принадлежность классу NP, необходимо доказать, что существует
верификатор такой, что при наличии действительного сертификата $c$ его можно проверить
за полиномиальное время. Пусть имеется верификатор для \textsc{HittingSet}, который для входа $(U, A, k, c)$, где $U$ --- множество, а $A$ --- множество подмножеств $U$, $k\in\mathbb{Z}^+$, переводит сертификат в некоторое множество, проверяет, что это множество действительно подмножество множества $U$, что его мощность не превышает $k$, что для каждого подмножества $A_i \in A$ выполнено $|A_i\cap X| = 1$. Все перечисленные операции могут быть проверены за время, полиномиально зависящее от входных параметров. Таким образом, задача \textsc{Hitting Set} лежит в $NP$.
\medskip\par 
Для доказательства, что \textsc{Hitting Set} является $NP$ полной, необходимо доказать, что уже известная $NP$-полная задача полиномиально сводима к \textsc{Hitting Set}. Возьмём известную $NP$-полную задачу \textsc{VertexCover}, которая говорит существует или нет вершинное покрытие мощности $k$. Пусть умеем решать задачу \textsc{VertexCover} для входных $(G, k)$, где $G$ --- некоторый граф, $k\in\mathbb{Z}^+$. Пусть множество вершин графа $G$ суть множество $U$ исходной задачи, а каждое ребро --- подмножество мощности $2$, содержащее элементы, соответствующие вершинам, которым оно инцидентно. Если в исходной задаче существует вершинное покрытие мощности $k$, то для каждого ребра выбрано ровно по одной вершине, ему инцидентной, что в терминах задачи \textsc{HittingSet} значит, что для каждого двухэлементного подмножества в множестве $X$ ровно один из элементов. В противном случае, если не нашлось вершинного покрытия мощности $k$, то для какого-либо ребра нельзя выбрать лишь одну из вершин, а, следовательно, для двухэлементного множества выбрать один элемент ему принадлежащий, который будет в множестве $X$.
\begin{flushright}
$\square$
\end{flushright}
\par
Теперь задача \textsc{HittingSet} для входа $(U, A, X, k)$ суть частный случай исходной задачи для $(U, A, X, C, k)$, где $C=\{c_i\}_{i=\overline{1, m}},$ $\forall\, i\, c_i=1$. Следовательно, исходная задача $NP$-сложна. Тем не менее, задача лежит в $NP$, так как сертификат может быть проверен за полиномиальное время.


\end{document}