\documentclass[10pt]{article}

%\pagestyle{plain}
\textwidth=16cm
\textheight=24.0cm
\oddsidemargin=0.3cm
\evensidemargin=0.5cm


\usepackage[utf8]{inputenc}
\usepackage[russian]{babel}
\usepackage{amssymb,amsmath}
\usepackage{fancyhdr}
\usepackage[yyyymmdd,hhmmss]{datetime}
\pagestyle{fancy}
\usepackage{graphicx}
\usepackage[a4paper,bmargin=2.5cm]{geometry}

\voffset=0pt
\headheight=0pt
\headsep=0pt



\begin{document}
\def\chap#1#2{\ \\ {\large\bf#1 \ | \ \tt\scshape#2} \par}
\pagestyle{empty}
{\bf
\small\centerline{Задание №3 --- <<NP-completeness>> \ | \ Лиана Хазалия, 3 курс 2 группа \ | \ \today, \currenttime}}
\ \\
\rm
{\large\texttt{Задача 99-1 [NP-Complectness]}}
\\
\textsc{Самый длинный путь}
\ \\[0.2cm]
\textit{Задан граф $G=(V,E)$ и положительное число $K\leq |V|$. Имеется ли в $G$ простой путь (то есть путь, не проходящий дважды ни через одну вершину), состоящий не менее чем из $K$ рёбер?}
\ \\[0.2cm]
\textbf{Решение 1.}  К данной задаче сведём задачу о \textsc{Гамильтоновом цикле}: дан граф и необходимо найти цикл, проходящий через каждую вершину по одному разу. Так как Гамильтонов цикл проходит по всем вершинам единожды, а также включает в себя любое из рёбер графа не более одного раза, то при нахождении Гамильтонова цикла находим и путь максимальной длины: $|V|-1$. То есть рассматриваемая задача содержит в качестве частного случая (при $K=|V|-1$) известную NP-полную задачу. Следовательно, является NP-полной.
\ \\[0.2cm]
\textbf{Решение 2.}  
 К данной задаче сведём задачу о \textsc{Гамильтоновом цикле}: дан граф и необходимо найти цикл, проходящий через каждую вершину по одному разу. Так как Гамильтонов цикл проходит по всем вершинам единожды, а также включает в себя любое из рёбер графа не более одного раза, то при нахождении Гамильтонова цикла находим и путь максимальной длины: $|V|-1$. То есть рассматриваемая задача содержит в качестве частного случая (при $K=|V|-1$) известную $NP$-полную задачу. Следовательно, является $NP$-полной.
\ \\[0.1 cm]
По существу описано выше, но скажем подробнее. Сведение принимает на вход $G(V, E)$ для задачи о гамильтонове цикле. Найдя гамильтонов цикл в графе $G$ ($|V|$ рёбер) получаем и простой путь длины $|V|-1$ удалением из найденногогамильтонова  цикла произвольного ребра. Данная процедура является сведением, так как в случае, если гамильтонов цикл найти удалось, удалим ребро и оставшаяся часть цикла --- простой путь длины $|V|-1$ в силу определения гамильтонова цикла, а в случае, если гамильтонова цикла нет, то покажем, что решений нет и у исходной задачи. Рассуждая от противного, предположим, что в графе $G$ есть простой путь длины $|V|-1$.

\ \\[0.5 cm]
{\large\texttt{Задача 100-10 [NP-Complectness]}}
\\
\textsc{Неэквивалентность регулярных выражений, не содержащих звёздочек}
\ \\[0.2cm]
\textit{Заданы два не содержащие звёздочек регулярных выражения $E_1$ и $E_2$ в конечном алфавите $\Sigma$. Такое выражение определяется следующим образом:
\begin{enumerate}
\item любой символ $\sigma$ алфавита $\Sigma$ есть не содержащее звёздочек регулярное выражение,
\item если $e_1$ и $e_2$ --- два не содержащие звёздочек регулярных выражения, то и слова $e_1e_2$ и $(e_1 \vee e_2)$ также не содержащие звёздочек регулярные выражения.
\end{enumerate}
Верно ли, что $E_1$ и $E_2$ представляют различные языки в алфавите $\Sigma$? (Язык, представляемый символом $\sigma\in\Sigma$, если $\{\sigma\}$, а если $e_1$ и $e_2$ представляют соответственно языки $L_1$ и $L_2$, то $e_1e_2$ представляет язык $\{xy:\,x\in L_1,\,y\in L_2\}$, а $(e_1\vee e_2)$ представляет язык $L_1\cup L_2$.)
}

\ \\[0.5 cm]
{\large\texttt{Задача 7 [NP-ISSUES]}}
\ \\[0.2cm]
\textit{
Так как задача о трёхмерном сочетании является $NP$-полной, естественно ожидать, что аналогичная задача о четырёхмерном сочетании будет хотя бы не менее сложной. Определим четырехмерное сочетание следующим образом: для заданных множеств $W$, $X$, $Y$ и $Z$, каждое из которых имеет размер $n$, и набора $C$ упорядоченных четверок в форме $(w_i, x_j, y_k , z_l)$ существуют ли $n$ четверок из $C$, среди которых никакие два не имеют общих элементов?
\\ \
Докажите, что задача о четырехмерном сочетании является $NP$-полной.
}
\ \\[0.2cm]
\textbf{Решение.} Пусть есть 4-дольный граф $G(V,E)$ на $4n$ вершинах (в каждой доли $n$ вершин), в котором можно выделить $m$ подграфов $K_4$. Далее отождествим каждую 4-клику, у которой нет двух вершин из одной доли, с новой вершиной, таким образом получим новый граф $G'$. Новые две вершины инцидентны тогда и только тогда, когда у подграфов, с которыми они были отождествлены, пересечение множеств вершин не пусто.
\ \\[0.1cm]
Если в таком новом графе $G'$ существует независимое множество вершин мощности $n$, то существует и решение у нашей задачи, потому что выделив $n$ вершин в $G'$, не имеющих общих ребер, выделили $n$ 4-клик исходного графа $G$, у которых нет общих элементов и у которых по одной вершине в каждой из четырёх доль в силу проведенного ранее отождествления. А значит, нашли решение исходной задачи. Если же независимого множества можности $n$ не существует, то какие бы $n$ 4-клик с вершинами в попарно различных долях ни были бы выбраны в исходном графе, какое-то из попарных пересечений множеств вершин не пусто, так как соответствующее множество вершин в $G'$ не является независимым.
\ \\[0.1cm]
По решению полученной задачи о поиске независимого множества в графе $G'$ легко восстановить разбиение исходного множества на четверки, элементы которых принадлежат попарно различным долям, а также нет общих.
\ \\[0.5 cm]
{\large\texttt{Задача 16 [NP-ISSUES]}}
\ \\[0.2cm]
\textit{Рассмотрим задачу характеристики множества по размерам его пересечений
с другими множествами. Имеется конечное множество $U$ размера $n$, а также
набор $A_1, \dots, A_m$ подмножеств $U$. Также заданы числа $c_1, \dots, c_m$. Вопрос звучит так: существует ли такое множество $X\subset U$, что для всех $i = 1, 2, \dots, m$ мощность $X \cap A_i$ равна $c_i$? Назовем его задачей выведения пересечений с входными данными $U$, $\{A_i\}$ и $\{c_i\}$.
\\ \
Докажите, что задача выведения пересечений является $NP$-полной.
}



\end{document}