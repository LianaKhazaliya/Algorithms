\documentclass[10pt]{article}

%\pagestyle{plain}
\textwidth=16cm
\textheight=24.0cm
\oddsidemargin=0.3cm
\evensidemargin=0.5cm


\usepackage[utf8]{inputenc}
\usepackage[russian]{babel}
\usepackage{amssymb,amsmath}
\usepackage{fancyhdr}
\usepackage[yyyymmdd,hhmmss]{datetime}
\usepackage{graphicx}
\usepackage[a4paper,bmargin=2.5cm]{geometry}

\voffset=0pt
\headheight=0pt
\headsep=0pt



\begin{document}
\def\chap#1#2{\ \\ {\large\bf#1 \ | \ \tt\scshape#2} \par}

{\bf
\begin{flushright}
\small{Задание №3 --- <<NP-completeness>> \\ Лиана Хазалия, 3 курс 2 группа\\  \today, \currenttime}
\end{flushright}}
\rm
\hline
\ \\[0.5 cm]
{\large\texttt{Задача 99-1 [NP-Complectness]}}
\\
\textsc{Самый длинный путь}
\ \\[0.1cm]
\textit{Задан граф $G=(V,E)$ и положительное число $K\leq |V|$. Имеется ли в $G$ простой путь (то есть путь, не проходящий дважды ни через одну вершину), состоящий не менее чем из $K$ рёбер?}
\ \\[0.3cm]
\textbf{Решение.}  
\medskip\par Так как длина гамильтонова пути равна $|V|-1$ (проходит по всем вершинам единожды, а также включает каждое из рёбер не более одного раза), то при нахождении гамильтонова пути решаем поставленную задачу для $K=|V|-1$. То есть известная задача UHamPath суть частный случай предложенной. Следовательно, предложенная задача LongestPath $NP$-сложна.
\medskip\par Рассматриваемая задача принадлежит $NP$, так как для заданного графа $G=(V, E)$ сертификат имеет вид последовательности, состоящей из вершин множества $V'\subseteq V$. При верификации проверяется, что в эту последовательность каждая вершина из $V'$ входит единожды и что каждая пара воследовательных вершин соединена ребром, то есть сертификат является действительным. Подобная проверка выполняется за полиномиальное время. 
\medskip\par Из того, что задача LongestPath $NP$-сложна и что LongestPath $\in$ $NP$, cледует её $NP$-полнота.
\ \\[0.5 cm]
\hline
\ \\[0.5 cm]
{\large\texttt{Задача 100-10 [NP-Complectness]}}
\\
\textsc{Неэквивалентность регулярных выражений, не содержащих звёздочек}
\ \\[0.1cm]
\textit{Заданы два не содержащие звёздочек регулярных выражения $E_1$ и $E_2$ в конечном алфавите $\Sigma$. Такое выражение определяется следующим образом:
\begin{enumerate}
\item любой символ $\sigma$ алфавита $\Sigma$ есть не содержащее звёздочек регулярное выражение,
\item если $e_1$ и $e_2$ --- два не содержащие звёздочек регулярных выражения, то и слова $e_1e_2$ и $(e_1 \vee e_2)$ также не содержащие звёздочек регулярные выражения.
\end{enumerate}
Верно ли, что $E_1$ и $E_2$ представляют различные языки в алфавите $\Sigma$? (Язык, представляемый символом $\sigma\in\Sigma$, если $\{\sigma\}$, а если $e_1$ и $e_2$ представляют соответственно языки $L_1$ и $L_2$, то $e_1e_2$ представляет язык $\{xy:\,x\in L_1,\,y\in L_2\}$, а $(e_1\vee e_2)$ представляет язык $L_1\cup L_2$.)
}
\ \\[0.3cm]
\textbf{Решение.}  
\medskip\par 
Докажем, что задача лежит в классе $NP$. Чтобы убедиться в неэквивалентности двух регулярных выражений, не содержащих звёздочек, должно быть найдено слово $x$, которое $x\in L_1$ и $x\not\in L_2$ или наоборот. Без нарушения общности, рассмотрим первый случай.
\medskip\par Нужно проверить, что $x\in L_1$ и что $x\not\in L_2$. Оба этих факт можно проверить сравнением слова $x$ с деревьями, которые соответствует регулярным выражениям $E_1$ и $E_2$, порождающим языки $L_1$ и $L_2$ соответственно.
\medskip\par Дерево регулярного выражения --- бинарное дерево, внутренним вершинам которого присвоена одна из операций дизъюнкции или конъюнкции, а вершинам-листьям присвоены символы алфавита $\Sigma^*=\Sigma \cup \{\emptyset\}$. Обходом дерева в обратном порядке можем сравнить слово $x$ с листьями каждого поддерева и узнать, какие префиксы слова $x$ каким поддеревьям могут соответствовать. По окончанию обхода дерева можно получить ответ: принадлежит слово $x$ языку, порожденному регулярным выражением, обход дерева которого был произведён, или нет.
\medskip\par Для представления регулярного выражения как бинарного дерева, а также для его обхода в обратном порядке, требуется линейное время. Предположим, что слово $x$ имеет длину $n$, тогда у $x$ можно различать $n+1$ префикс. Следовательно, асимпотика алгоритма установления принадлежности слова $x$ языку $L_1$ равна $\mathcal{O}\left(n\cdot max\left(|E_1|,|E_2|\right)\right)$. В данном случае, важную роль играет то, что регулярные выражения $E_1$ и $E_2$ не содержат звёздочек Клини, что позволяет говорить о том, что высота дерева ограничена длиной регулярного выражения, ему соответствующего: в листьях символы алфавита $\Sigma^*$, дизъюнкция не увеличивает длину строки, а конъюнкция складывает длины, --- длина регулярного выражения $n\leq max\left(|E_1|,|E_2|\right)$ и верификация выполняется за полиномиальное время.




\ \\[0.5 cm]
\hline
\ \\[0.5 cm]
{\large\texttt{Задача 7 [NP-ISSUES]}}
\ \\[0.1cm]
\textit{
Так как задача о трёхмерном сочетании является $NP$-полной, естественно ожидать, что аналогичная задача о четырёхмерном сочетании будет хотя бы не менее сложной. Определим четырехмерное сочетание следующим образом: для заданных множеств $W$, $X$, $Y$ и $Z$, каждое из которых имеет размер $n$, и набора $C$ упорядоченных четверок в форме $(w_i, x_j, y_k , z_l)$ существуют ли $n$ четверок из $C$, среди которых никакие два не имеют общих элементов?
\\ \
Докажите, что задача о четырехмерном сочетании является $NP$-полной.
}
\ \\[0.3cm]
\textbf{Решение.} 
\medskip\par 
Покажем, что редложенная задача 4-dim matching $NP$-сложна, доказав 3-dim matching $\leq_P$ 4-dim matching. Сведение начинается с экземпляра задачи 3-dim matching. Пусть дано множество  $M_3 \subseteq W\times X\times Y$, где $W$, $X$ и $Y$ --- непересекающиеся множества, содержащие одинаковое число элементов $q$. Четвертое множество $Z$ мощности $q$ берется так, чтобы пересечение с любым из уже имеющихся было пусто. Задаём некоторое биективное соответствие между множествами $Y$ и $Z$. Каждая четверка --- существовавшая ранее тройка, к которой добавлена четвёртая координата, которой присвоено значение элемента множества $Z$, который соответствует элементу множества $Y$ в третьей координате взятой тройки в ранее заданной биекции.
\medskip\par Покажем, что проведенное преобразование является сведением. Допустим, что для множества четвёрок $M_4\subseteq W\cup X\cup Y\cup Z$ существует такое подмножество $M'_4\subseteq M_4$, что $|M'_4|=q$ и что любой элемент множества $W\cup X\cup Y\cup Z$ принадлежит ровно одной из четверок множества $M'_4$. Тогда верно и утверждение, что любой элемент множества $W\cup X\cup Y$ принадлежит ровно одной из троек множества $M'_3\subseteq M_3$, которое получено удалением четвертой координаты в каждой четверке. Проведём и обратные рассуждения. Предположим, что нет решения у задачи 4-dim matching. Тогда как минимум один из элементов $W\cup X\cup Y$ не покрыт или покрыт с повторами, так как не покрыть (или покрыть с повторами) элементы множества $Z$ значит не покрыть (или покрыть с повторами) и элементы множества $Y$, потому что между ними имеется биекция.
\medskip\par Чтобы показать, что задача 4-dim matching лежит в $NP$, воспользуемся множеством $M_4\subseteq W\cup X\cup Y\cup Z$ в качестве сертификата. Проверить, лежит ли каждый элемент из множества $W\cup X\cup Y\cup Z$ в ровно одной из четверок множества $M_4$ можно в течение полиномиального времени.
\ \\[0.5 cm]
\hline
\ \\[0.5 cm]
{\large\texttt{Задача 16 [NP-ISSUES]}}
\ \\[0.1cm]
\textit{Рассмотрим задачу характеристики множества по размерам его пересечений
с другими множествами. Имеется конечное множество $U$ размера $n$, а также
набор $A_1, \dots, A_m$ подмножеств $U$. Также заданы числа $c_1, \dots, c_m$. Вопрос звучит так: существует ли такое множество $X\subset U$, что для всех $i = 1, 2, \dots, m$ мощность $X \cap A_i$ равна $c_i$? Назовем его задачей выведения пересечений с входными данными $U$, $\{A_i\}$ и $\{c_i\}$.
\\ \
Докажите, что задача выведения пересечений является $NP$-полной.
}
\ \\[0.3cm]
\textbf{Решение.} 
\medskip\par 
Рассмотрим задачу 3-dim Mathing: дано множество  $M \subseteq W\times X\times Y$, где $W$, $X$ и $Y$ --- непересекающиеся множества, содержащие одинаковое число элементов $q$. Верно ли, что $M$ содержит трёхмерное сочетание, то есть подмножество $M'\subseteq M$, такое, что $|M'|=q$ и никакие два разных элемента $M'$ не имеют ни одной равной координаты?
\medskip\par
Сначала докажем, что задача $NP$-сложна по причине сводимости к ней 3-dim Matching. Пусть дан экземпляр задачи 3-dim Matching: некоторые непересекающиеся множества $W=\{w_i\}_{i=1}^q$, $X=\{x_i\}_{i=1}^q$, $Y=\{y_i\}_{i=1}^q$ мощности $q\in\mathbb{N}$ и некоторое множество $M\subseteq W\times X\times Y$. Пусть множество $U$ исходной задачи совпадает с множеством троек $M$. Далее, зададим множества подмножеств $\{A_i\}_{i=1}^{3n}$ множества $U$ следующим образом: 
\begin{align*}
\left\{A_{i}\right\}_{i=1}^q &=  \left\{A_i=\left\{(w_i, x, y)|x\in X,\, y\in Y\right\}\right\}_{i=1}^q,\\
\left\{A_{q+i}\right\}_{i=1}^q &= \left\{A_{q+i}=\left\{(w, x_i, y)|w\in W,\, y\in Y\right\}\right\}_{i=1}^q,\\
\left\{A_{2q+i}\right\}_{i=1}^q &= \left\{A_{2q+i}=\left\{(w, x, y_i)|w\in W,\, x\in X\right\}\right\}_{i=1}^q.
\end{align*}
\par То есть в подмножество, например, $A_1$ входят все тройки, у которых в первой координате значение равно значению элемента $w_1$ множества $W$. 
\medskip\par
Определим ещё одно подмножество $A_{3q+1}=U$. Множество параметров $\{c_i\}_{i=1}^{3q+1}$ зададим как $c_i=1$ $\forall\,i=\overline{1, 3q}$, а $c_{3q+1}=q$.
Также ст\'{o}ит отметить, что для создания экземпляра рассматриваемой задачи из экземпляра известной (3-dim matching) требуется время $\mathcal{O}(q^3)$.
\medskip\par 
Такое преобразование является сведением, так как если множество $X\subseteq U$ для сформулированной задачи выведения пересечений с описанными ранее входными данными $U$, $\{A_i\}$ и $\{c_i\}$ существует, то взято ровно $q$ троек, так как $c_{3q+1}=|A_{3q+1}\cap X|= |U\cap X| = |M \cap X| = q$. Также каждый элемент из множества $W\cup X\cup Y$ в $X$ встречается единожды, в силу того что мощность пересечения $|X \cap A_i| = |X \cap \{\text{все тройки с фиксированным элементом}\}| = c_i = 1$ для любого $i=\overline{1,3q}$. То есть множество $X$ является искомым множеством $M'$ для задачи 3-dim matching. Если же решения сформулированной задачи не существует, то по крайней мере одно ограничение не было выполнено, следовательно, по крайней мере один элемент из $W\cup X\cup Y$ не был покрыт или какой-то учтен несколько раз. Таким образом, сведение верно.
\medskip\par 
Чтобы доказать, что принадлежность классу $NP$, необходимо доказать, что существует
верификатор такой, что при наличии действительного сертификата его можно проверить
за полиномиальное время. Пусть имеется верификатор, который для входа $(U, \{A_i\},\{c_i\})$ проверяет, что некоторое множество $X$ (сертификат) является подмножеством множества $U$, что для каждого подмножества $A_i \in A$ выполнено $|A_i\cap X| = c_i$. Все перечисленные операции могут быть проверены за время, полиномиально зависящее от входных параметров. Таким образом, рассматриваемая задача выведения пересечений лежит в $NP$.
\medskip\par 
Таким образом, исходная задача $NP$-полна, потому что была доказана возможность полиномиального сведения известной $NP$-полной задачи к рассматриваемой, лежащей в $NP$. 
\ \\[0.2 cm]
\hline


\end{document}