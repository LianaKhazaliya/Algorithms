\documentclass[10pt]{article}

%\pagestyle{plain}
\textwidth=16cm
\textheight=24.0cm
\oddsidemargin=0.3cm
\evensidemargin=0.5cm


\usepackage[utf8]{inputenc}
\usepackage[russian]{babel}
\usepackage{amssymb,amsmath}
\usepackage{fancyhdr}
\usepackage[yyyymmdd,hhmmss]{datetime}
\usepackage{graphicx}
\usepackage[a4paper,bmargin=2.5cm]{geometry}

\voffset=0pt
\headheight=0pt
\headsep=0pt



\begin{document}
\def\chap#1#2{\ \\ {\large\bf#1 \ | \ \tt\scshape#2} \par}

{\bf
\begin{flushright}
\small{Задание №3 --- <<NP-completeness>> \\ Лиана Хазалия, 3 курс 2 группа\\  \today, \currenttime}
\end{flushright}}
\ \\
\rm
{\large\texttt{Задача 99-1 [NP-Complectness]}}
\\
\textsc{Самый длинный путь}
\ \\[0.2cm]
\textit{Задан граф $G=(V,E)$ и положительное число $K\leq |V|$. Имеется ли в $G$ простой путь (то есть путь, не проходящий дважды ни через одну вершину), состоящий не менее чем из $K$ рёбер?}
\ \\[0.2cm]
\textbf{Решение 1.}  Пусть дан граф и необходимо найти цикл, проходящий через каждую вершину по одному разу. Так как Гамильтонов цикл проходит по всем вершинам единожды, а также включает в себя любое из рёбер графа не более одного раза, то при нахождении Гамильтонова цикла находим и путь максимальной длины: $|V|-1$. То есть рассматриваемая задача содержит в качестве частного случая (при $K=|V|-1$) известную NP-полную задачу. Следовательно, является NP-сложной. Однако задача лежит в $NP$, а, следовательно, $NP$-полна, так как проверяющую функцию для сертификата, который для данной задачи суть гамильтонов путь в графе, работает за полиномиальное время.
\ \\[0.2cm]
\textbf{Решение 2. (сведение)}  
 К данной задаче сведём задачу о \textsc{Гамильтоновом пути} в ориентированном графе. Пусть дан ориентированный граф $G$, по нему строим новый неориентированный граф $G'$: а именно (1) каждую вершину графа $G$ отождествим с тремя новыми в графе $G'$, причём эти три образуют цепь, у которой отличаем вершину-начало и вершину-конец, (2) каждом ориентированному ребру графа $G$ сопоставим ребро такое, что оно исходит из вершины-конца и входит в вершину-начало соответствующих вершинам исходного графа цепей. Таким образом, если в полученном графе $G'$ имеется гамильтонов путь, то и в ориентированном графе $G$ имеется ориентированный гамильтонов путь. И наоборот, если пути нет в $G'$, то и в $G$ ориентированного пути не существует. Цепи в графе $G'$ и такое их соединение искусственно переносят структуру ориентированного графа на неориентированный.
\medskip\par
Теперь докажем, что задача о гамильтоновом пути в ориентированном графе $NP$-полна. Сведём к неё задачу об ориентированном гамильтоновом цикле. Пусть дан ориентированный граф $G(V, E)$, по нему построим новый граф $G'(V', E')$: выберем какую-либо из вершин $\vartheta \in V$ графа $G$, в $G'$ добавим ещё одну вершину $\upsilon$, а далее распределим ребра, инцидентные вершине $\vartheta$ графа $G$ так, что вершине $\vartheta'$ графа $G'$ инцедентны только исходящие ребра вершины $\vartheta$, а все входящие с сохренением всех инцедентностей вершинам в графе $G'$ входят в вершину $\upsilon$.
\medskip\par 
Так если в $G'$ был ориентированный гамильтонов цикл, то в $G'$, очевидно, будет ориентированный гамильтонов путь. Если в полученном графе $G'$ есть ориентированный гамильтонов путь, то в графе $G$ имеется ориентированный гамильтонов цикл, потому что путь в $G'$ имеет начало в $\vartheta'$, а конец в $\upsilon$, а в графе $G$ ребра обеих инцедентны $\vartheta$. Задача о поиске ориентированного цикла в качестве подзадачи содержит $NP$ полную о нахождении цикла в неориентированном графе, следовательно, $NP$-сложна. Но сертификат (ориентированный гамильтонов цикл в $G$) можно проверить за полиномиальное время. Отсюда она лежит в $NPC$.
\medskip\par 
Так доказано, что исходная задача $NP$-сложна. Однако задача лежит в $NP$, а, следовательно, $NP$-полна, так как проверяющую функцию для сертификата, который для данной задачи суть гамильтонов путь в графе, работает за полиномиальное время.
\ \\[0.5 cm]
{\large\texttt{Задача 100-10 [NP-Complectness]}}
\\
\textsc{Неэквивалентность регулярных выражений, не содержащих звёздочек}
\ \\[0.2cm]
\textit{Заданы два не содержащие звёздочек регулярных выражения $E_1$ и $E_2$ в конечном алфавите $\Sigma$. Такое выражение определяется следующим образом:
\begin{enumerate}
\item любой символ $\sigma$ алфавита $\Sigma$ есть не содержащее звёздочек регулярное выражение,
\item если $e_1$ и $e_2$ --- два не содержащие звёздочек регулярных выражения, то и слова $e_1e_2$ и $(e_1 \vee e_2)$ также не содержащие звёздочек регулярные выражения.
\end{enumerate}
Верно ли, что $E_1$ и $E_2$ представляют различные языки в алфавите $\Sigma$? (Язык, представляемый символом $\sigma\in\Sigma$, если $\{\sigma\}$, а если $e_1$ и $e_2$ представляют соответственно языки $L_1$ и $L_2$, то $e_1e_2$ представляет язык $\{xy:\,x\in L_1,\,y\in L_2\}$, а $(e_1\vee e_2)$ представляет язык $L_1\cup L_2$.)
}

\ \\[0.5 cm]
{\large\texttt{Задача 7 [NP-ISSUES]}}
\ \\[0.2cm]
\textit{
Так как задача о трёхмерном сочетании является $NP$-полной, естественно ожидать, что аналогичная задача о четырёхмерном сочетании будет хотя бы не менее сложной. Определим четырехмерное сочетание следующим образом: для заданных множеств $W$, $X$, $Y$ и $Z$, каждое из которых имеет размер $n$, и набора $C$ упорядоченных четверок в форме $(w_i, x_j, y_k , z_l)$ существуют ли $n$ четверок из $C$, среди которых никакие два не имеют общих элементов?
\\ \
Докажите, что задача о четырехмерном сочетании является $NP$-полной.
}
\ \\[0.2cm]
\textbf{Решение. (не оттуда, не туда)} Пусть есть 4-дольный граф $G(V,E)$ на $4n$ вершинах (в каждой доли $n$ вершин), в котором можно выделить $m$ подграфов $K_4$. Далее отождествим каждую 4-клику, у которой нет двух вершин из одной доли, с новой вершиной, таким образом получим новый граф $G'$. Новые две вершины инцидентны тогда и только тогда, когда у подграфов, с которыми они были отождествлены, пересечение множеств вершин не пусто.
\medskip\par
Если в таком новом графе $G'$ существует независимое множество вершин мощности $n$, то существует и решение у нашей задачи, потому что выделив $n$ вершин в $G'$, не имеющих общих ребер, выделили $n$ 4-клик исходного графа $G$, у которых нет общих элементов и у которых по одной вершине в каждой из четырёх доль в силу проведенного ранее отождествления. А значит, нашли решение исходной задачи. Если же независимого множества можности $n$ не существует, то какие бы $n$ 4-клик с вершинами в попарно различных долях ни были бы выбраны в исходном графе, какое-то из попарных пересечений множеств вершин не пусто, так как соответствующее множество вершин в $G'$ не является независимым.
\ \\[0.1cm]
По решению полученной задачи о поиске независимого множества в графе $G'$ легко восстановить разбиение исходного множества на четверки, элементы которых принадлежат попарно различным долям, а также нет общих.
\ \\[0.5 cm]
{\large\texttt{Задача 16 [NP-ISSUES]}}
\ \\[0.2cm]
\textit{Рассмотрим задачу характеристики множества по размерам его пересечений
с другими множествами. Имеется конечное множество $U$ размера $n$, а также
набор $A_1, \dots, A_m$ подмножеств $U$. Также заданы числа $c_1, \dots, c_m$. Вопрос звучит так: существует ли такое множество $X\subset U$, что для всех $i = 1, 2, \dots, m$ мощность $X \cap A_i$ равна $c_i$? Назовем его задачей выведения пересечений с входными данными $U$, $\{A_i\}$ и $\{c_i\}$.
\\ \
Докажите, что задача выведения пересечений является $NP$-полной.
}
\ \\[0.2cm]
\textbf{Решение.} К данной задаче сведём 3-SAT. Пусть даны литералы $x_1, \dots, x_n$. Далее имеем список дизъюнктов, каждый из которых содержит по $3$ литерала. Каждому литералу и его отрицанию поставим в соответствие элементы множества $U$ некоторой инъективной функцией $f$. Пусть количество дизъюнктов в данном списке равно $m$, тогда в рассматриваемой задаче имеем $m$ подмножеств $\{A_i\}_{i=\overline{1,m}}$ множества $U$, где для всякого $i=\overline{1, m}$ множество $A_i$ состоит ровно из $3$ элементов, соответствующих литералам в $i$-ом дизъюнкте.
\medskip\par
Пусть существуют такие значения литералов, для которых коньюнкция исходных дизъюнктов истинна. Будем считать, что такой набор задаёт множество $X$, пересечения подмножеств с которым рассматриваем: элемент $f(x_i)\in X$, если $x_i=1$, в случае $x_i=0$ элемент $f(\overline{x_i})\in X$. Число $c_i=1$ для множества $A_i$ суть количество литералов, обращающихся в истину при подстановке значений, в $i$-ом дизъюнкте.
\medskip\par
Теперь, если для исходного набора дизъюнктов существует набор значений литералов, при котором формула выполнена, то и для полученного набора подмножеств $\{A_i\}_{i=\overline{1,m}}$ существует такое множество $X\subset U$, что для всех $i = 1, 2, \dots, m$ мощность $X \cap A_i$ равна $c_i$, где $c_i$ --- количество литералов, которые обращаются в истину в $i$-ом дизъюнкте ($1\leq c_i \leq 3$). Если же исходная формула тождественно равна нулю, то нельзя выбрать множество $X$ так, чтобы пересечение с каким-то из $A_i$ было мощности $c_i$, где $c_i$. То если наличие решения у 3-SAT свидетельствует о наличии решения поставленной при $|A_i|=3$ для $i=\overline{1, m}$ и какого-то набора $c_i$ ($c_i\in\mathbb{Z}^+$).


\end{document}